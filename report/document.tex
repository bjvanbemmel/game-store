\documentclass[a4paper]{report}
\usepackage[dutch]{babel}
\usepackage[backend=biber,style=apa]{biblatex}
\addbibresource{citations.bib}
\usepackage{csquotes}
\usepackage{lmodern}
\renewcommand{\familydefault}{\sfdefault}

\title{Front-end keuzedeel eindverslag}
\author{Beau Jean van Bemmel, 97085143}
\begin{document}
    \maketitle

    \chapter*{Voorwoord}
    \pagenumbering{gobble}
    Dit verslag gaat ervan uit dat de lezer al wat basis kennis heeft op het gebied van web development.
    Zo wordt er hier en daar vakterminologie gebruikt. Bij elke eerste instantie van zo'n woord is er ook een referentie te vinden naar een externe bron.
    Aan het einde van dit document vindt u ook een bibliografie met alle gebruikte bronnen.

    Ik hoop met dit verslag wat inspiratie te kunnen creëren bij de lezer(s) om ook op pad te gaan en een nieuw framework te leren en toe te passen.
    Er is niets belangrijker voor een software developer dan de vaardigheid om uit zijn of haar comfort-zone te kunnen kruipen.
    \tableofcontents

    \chapter{Inleiding}
    \pagenumbering{arabic}
    De afgelopen twéé maanden ben ik bezig geweest met het \textit{front-end development} keuzedeel.
    Gedurende deze periode heb ik o.a. veel kennis opgedaan van 
    \textit{Nuxt}\footfullcite{nuxt}, een 
    \textit{meta-framework}\footfullcite{meta} voor 
    \textit{Vue.js}\footfullcite{vue}.
    
    Naast het gebruiken van een front-end framework heb ik er ook voor gekozen om een \textit{Application Programming Interface (API)}\footfullcite{api} te bouwen en gebruiken.
    Deze API heb ik gemaakt met \textit{Go}\footfullcite{go}. Meer over deze beslissing en de uitwerking hiervan vindt u in een later hoofdstuk.

    \chapter{Gebruikte technologieën}
    \section{TypeScript}

    \section{Go}

    \section{Nuxt}
    Ik had voor Nuxt gekozen, omdat dit framework met \textit{Server Side Rendering (SSR)} komt.
    Dit houdt in dat de HTML op de server gegenereerd wordt i.p.v. op de client. De aanwezigheid van SSR maakt het bot vriendelijk - om het zo even te noemen.

    Nu maakt Nuxt niet alleen maar gebruik van SSR. Feitelijk maakt het gebruik van \textit{Universal Rendering}\footfullcite{universal}.
    Dat wil betekenen dat op de \textit{initial page load} - het moment dat de pagina voor het eerst wordt ingeladen - alle HTML gegenereerd wordt door de server
    en verstuurd wordt naar de client. Naast deze HTML wordt ook een lading \textit{JavaScript (JS)} meegestuurd. Deze JS neemt het hierna over en maakt van de applicatie een
    \textit{Single-page application (SPA)}\footfullcite{spa}.

    \section{Tailwind}

    \section{Docker}

    \chapter{SEO}

    \chapter{Toegankelijkheid}

    \chapter{Verbetervoorstellen}

    \chapter{Reflectie}

    \printbibliography
\end{document}
