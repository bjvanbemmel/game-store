\documentclass[a4paper]{report}
\usepackage[dutch]{babel}
\usepackage[backend=biber,style=apa]{biblatex}
\addbibresource{citations.bib}
\usepackage{csquotes}
\usepackage{lmodern}
\usepackage{listings}
\lstdefinestyle{css}{%
    xleftmargin={0.75cm},
    numbers=left,
    stepnumber=1,
    firstnumber=1,
    numberfirstline=true,
    tabsize=4,
    showtabs=false,
    showspaces=false,
    showstringspaces=false,
    extendedchars=true,
    breaklines=true,
    frame=single,
}
\renewcommand{\familydefault}{\sfdefault}

\title{Front-end keuzedeel eindverslag}
\author{Beau Jean van Bemmel, 97085143}
\begin{document}
    \maketitle

    \chapter*{Voorwoord}
    \pagenumbering{gobble}
    Dit verslag gaat ervan uit dat de lezer al wat basis kennis heeft op het gebied van web development.
    Zo wordt er hier en daar vakterminologie gebruikt. Bij elke eerste instantie van zo'n woord is er ook een referentie te vinden naar een externe bron.
    Aan het einde van dit document vindt u ook een bibliografie met alle gebruikte bronnen.

    Ik hoop met dit verslag wat inspiratie te kunnen creëren bij de lezer(s) om ook op pad te gaan en een nieuw framework te leren en toe te passen.
    Er is niets belangrijker voor een software developer dan de vaardigheid om uit zijn of haar comfort-zone te kunnen kruipen.
    \tableofcontents

    \chapter{Inleiding}
    \pagenumbering{arabic}
    De afgelopen twéé maanden ben ik bezig geweest met het \textit{front-end development} keuzedeel.
    Gedurende deze periode heb ik o.a. veel kennis opgedaan van 
    \textit{Nuxt}\footfullcite{nuxt}, een 
    \textit{meta-framework}\footfullcite{meta} voor 
    \textit{Vue.js}\footfullcite{vue}.
    
    Naast het gebruiken van een front-end framework heb ik er ook voor gekozen om een \textit{Application Programming Interface (API)}\footfullcite{api} te bouwen en gebruiken.
    Deze API heb ik gemaakt met \textit{Go}\footfullcite{go}. Meer over deze beslissing en de uitwerking hiervan vindt u in een later hoofdstuk.

    In de volgende hoofdstukken vertel ik u graag over de keuzes die ik heb moeten maken, de situaties waarin ik vast liep en de ontdekkingen die ik heb gemaakt.

    \chapter{Gebruikte technologieën}
    Voordat we beginnen met de SEO en toegankelijkheidsrapportages wil ik even een moment nemen om mijn \textit{tech stack} te introduceren.
    Dit houdt in de talen, frameworks, libraries en diverse tools waar ik gebruik van heb gemaakt gedurende de ontwikkeling van dit keuzedeel:

    \section{TypeScript}
    Om te beginnen wil ik de meest belangrijke taal in deze \textit{stack} introduceren: \textit{TypeScript}\footfullcite{ts}.
    Aan het begin van dit keuzedeel heb ik de keuze gemaakt om in plaats van het normale \textit{JavaScript} gebruik te maken van TypeScript.
    De reden hiervoor is omdat ik een groot fan ben van types. Het maakt code veel voorspelbaarder en ook makkelijker te lezen.
    Een ander groot pluspunt is het feit dat de \textit{Language Server Protocol (LSP)}\footfullcite{lsp} enorm fijn meewerkt met een typed taal.

    De laatste maanden ben ik veel bezig geweest met PHP (in combinatie met Laravel), C\# en Go. Één ding dat al deze talen en frameworks met elkaar gemeen hebben
    is dat ze gebruik maken van types. PHP heeft tegenwoordig strict typing. Sinds dit is uitgekomen maakt Laravel gebruik van strict typing op alle APIs die in aanraking komen met de developer.
    C\# en Go zijn ook beiden typed talen. Geen hebben een optie om dynamisch te werken (enige vergelijkbare punt is dat ze beiden \textit{type inference} hebben, dus hoef je niet alles zelf aan te geven).

    Sinds ik gewend ben geraakt aan het werken met strict typing vond ik het een wijze keuze om dit ook in het JavaScript ecosysteem toe te passen met TypeScript.
    Met mijn bestaande kennis van \textit{interfaces} en \textit{enums} heb ik veel leuke trucjes uit weten te voeren om solide, veilige en toch uitbreidbare code te schrijven.

    \section{Go}
    Aan het begin van dit keuzedeel heb ik nog een andere bijzondere beslissing gemaakt: Het bouwen van een API.
    Het klinkt misschien gek om voor een front-end keuzedeel ook de focus te gooien op back-end, maar de reden hiervoor was uiteindelijk heel simpel:
    Ik wilde graag een dynamische website maken die ook met zoekopdrachten zou werken. Hiervoor kun ik uiteraard gewoon een JSON mokken en uiteindelijk via JS op arrays filteren,
    maar ik vond het leuker om echt te werken met een API.
    Ook vond ik het een mooie kans om een API te bouwen met Go, sinds ik dat nog niet vaak gedaan had. Wanneer je de code van dit project ziet, dan lijkt het wel alsof de focus
    op de back-end lag. Dit komt omdat de root van dit project eigenlijk een Go project-structuur heeft. Het Nuxt project bevindt zich in een sub-directory genaamd ''web``.
    Nu moet ik toegeven dat ik deze back-end drie keer opnieuw heb geschreven. Uiteraard betekent dit dat ik veel heb geleerd op niet alleen de front-end kant, maar ook de back-end.
    Achteraf blijkt het dus wel een goede beslissing te zijn om een back-end te hebben gebouwd met Go. Dit heeft de front-end ook mooier gemaakt, omdat ik zoekfuncties en laad-animaties heb kunnen toevoegen.

    \section{Nuxt}
    Ik had voor Nuxt gekozen, omdat dit framework met \textit{Server Side Rendering (SSR)} komt.
    Dit houdt in dat de HTML op de server gegenereerd wordt i.p.v. op de client. De aanwezigheid van SSR maakt het bot vriendelijk - om het zo even te noemen.

    Nu maakt Nuxt niet alleen maar gebruik van SSR. Feitelijk maakt het gebruik van \textit{Universal Rendering}\footfullcite{universal}.
    Dat wil betekenen dat op de \textit{initial page load} - het moment dat de pagina voor het eerst wordt ingeladen - alle HTML gegenereerd wordt door de server
    en verstuurd wordt naar de client. Naast deze HTML wordt ook een lading \textit{JavaScript (JS)} meegestuurd. Deze JS neemt het hierna over en maakt van de applicatie een
    \textit{Single-page application (SPA)}\footfullcite{spa}.

    Nuxt maakt gebruik van de Vue.js \textit{templating engine}. Ik had al ervaring met Vue.js, dus was het niet enorm lastig om te wennen aan Nuxt.
    Door Nuxt te gebruiken ontvang je allerlei ingebouwde functionaliteiten, zoals:
    \textit{State Management}\footfullcite{state},
    \textit{Automatic Imports}\footfullcite{autoimports},
    \textit{File-based Routing}\footfullcite{routing}
    en andere functies waar ik geen gebruik van heb gemaakt.

    \section{Tailwind}
    Bij een front-end opdracht hoort HTML, en bij HTML hoort CSS.
    Al hoewel ik voldoende ervaring heb met het schrijven van CSS, vind ik het een vervelende taal om mee te werken.
    Het creëert een gigantische lijst aan korte \textit{key:value} pairs, die allemaal gelinkt zijn aan een element, class of id - soms ook met speciale selectors.
    Zo'n CSS bestand gaat al snel de duizenden lijnen in bij een applicatie van gemiddelde grootte. Nu biedt Vue.js (en daarbij dus ook Nuxt) de mogelijkheid aan om
    \textit{scoped CSS}\footfullcite{scopedcss} te gebruiken. Helaas verhelpt dit niet het probleem dat je een gigantische lijst met korte lijnen krijgt.
    Uiteraard kun je de CSS minimaliseren, maar dan wordt het nog onleesbaarder en lastiger te navigeren voor de developer.

    Van dit probleem hebben meerdere mensen last, daarom bestaat er een CSS library genaamd \textit{Tailwind CSS}\footfullcite{tailwind}.

    Het idee van Tailwind is dat alle mogelijke styling options een eigen class hebben. Wat volgt is een klein voorbeeld van vanilla CSS vergeleken met Tailwind classes.
    \begin{lstlisting}[caption={Een Flexbox container met een zwarte rand in vanilla CSS}, label=useless, style=css]
    .container {
        display: flex;
        justify-content: center;
        align-items: center;
        border: .1rem solid black;
    }
    \end{lstlisting}
    \begin{lstlisting}[caption={Dezelfde container geschreven in Tailwind}, label=useless, style=css]
    <div
        class="flex justify-center items-center border-1 border-black"
    ></div>
    \end{lstlisting}

    In het begin zal dit misschien als een slechte oplossing lijken. Zo creëer je namelijk i.p.v. een gigantische lijst aan korte lijnen een grote lijst met classes.
    Dit klopt in principe ook, maar in de praktijk zult u zien dat het een verbetering is op standaard CSS. Alle styling wordt gedefinieerd exact waar het nodig is.
    Zo hoef je dus niet steeds twee bestanden met elkaar te vergelijken. Hierdoor is het in één oogopslag duidelijk hoe een HTML element eruit komt te zien.

    Tailwind bevat ook predefined classes voor kleuren, border-radius, shadow etc. Zo kunt u bijvoorbeeld {\fontfamily{courier}\selectfont bg-red-500} gebruiken
    om een hoog contrast rood te krijgen. Tailwind kent dus een groot hoeveelheid classes die gecureerde styling bevatten. Niet op het niveau als
    \textit{Bootstrap}\footfullcite{bootstrap}, waar kant en klare buttons bestaan, maar op vorm en kleur niveau. Dit geeft een developer de flexibiliteit van vanilla CSS
    met het gemak van Bootstrap.

    \section{Docker}
    Ik codeer op twéé verschillende apparaten: mijn laptop en desktop PC. Welk apparaat ik gebruik hangt af van mijn locatie.
    Nu kunt u zich misschien wel voorstellen dat ik het zo makkelijk mogelijk wil maken voor mijzelf om te kunnen schakelen tussen deze twéé apparaten.
    Daarom heb ik ervoor gekozen om Docker te gebruiken voor dit project. Feitelijk gebruik ik Docker voor al mijn projecten, dus was dit geen lastige beslissing om te maken.

    Docker geeft developers de mogelijkheid om op elk apparaat een identieke ontwikkelomgeving neer te zetten. Het is vergelijkbaar met \textit{virtual machines (VMs)}, maar dan
    vele malen lichter. Het grootste verschil tussen de twéé ligt hem in het feit dat VMs de hardware emuleren, waarals Docker het operatie systeem emuleert. De kernels worden
    verder dus nog wel gedeeld tussen de host en de Docker containers.

    Voor de meeste situaties is dit niveau van virtualisatie meer dan voldoende, daarom heeft bijna de gehele industrie nu al kennis gemaakt met cloud software zoals Docker.
    Binnen de code van dit project vindt u een aantal bestanden, zoals {\fontfamily{courier}\selectfont docker-compose.yml, docker/go/Dockerfile} en
    {\fontfamily{courier}\selectfont docker/node/Dockerfile}. Deze bestanden zorgen ervoor dat ik met een enkel commando mijn database, front-end en back-end op kan starten op welke omgeving dan ook.

    Docker is puur gemak, dus maak ik er graag gebruik van.

    \chapter{Lighthouse rapportage}
    Wat volgt is een rapportage van \textit{Lighthouse}\footfullcite{lighthouse} op het gebied van \textit{Search Engine Optimization (SEO)}\footfullcite{seo},
    \textit{performance}, toegankelijkheid en \textit{best practices}.
    Ik noteer de punten die ik van Lighthouse krijg, licht toe wat ermee wordt bedoeld en hoe ik het mogelijk zou kunnen verbeteren.

    \section{Performance}
    Op het gebied van \textit{performance} scoort de website een \textbf{96}.
    Dit is met grote dank aan de snelheid van Nuxt, dus neem ik hier niet teveel credit voor.
    Lighthouse geeft mij wel een aantal \textit{opportunities} - verbeterpunten, waarmee de website uiteindelijk \textbf{100} punten kan krijgen.
    Helaas zijn een aantal van deze verbeterpunten moeilijk toe te passen.

    \subsection{Properly size images}
    Volgens Lighthouse kunnen wij ongeveer 3.77 secondes aan laadtijd besparen door de bestandsgrootte van de afbeeldingen kleiner te maken.
    Al hoewel de 3.77 secondes ruim ingeschat zijn, gezien het feit dat de pagina laadt in onder een seconde tijd, ben ik het hier wel mee eens.

    Momenteel worden alle afbeeldingen gehaald van drie bronnen: de \textit{Content Delivery Network (CDN)}\footfullcite{cdn} van DuckDuckGo,
    de CDN van Cloudflare en de CDN van Akamai.

    Deze CDNs zijn van nature al enorm snel in het ophalen van afbeeldingen, maar de bestandsgroottes zijn uiteraard niet geoptimaliseerd voor mijn gebruik.
    Een mogelijke optie is dan ook om zelf alle afbeeldingen te downloaden, ze te compressen en ze als laatste te croppen naar de verschillende formaten waarin ik ze gebruik.
    Aan het begin van dit project was er al een idee om een eigen CDN op te bouwen, maar die heb ik er later uitgehaald omdat er te veel focus kwam te liggen op de back-end.

    \subsection{Serve images in next-gen formats}
    Dit sluit een beetje aan bij het vorige verbeterpunt. In plaats van bestandsgrootte (ook al wordt dit er uiteraard ook in meegenomen) gaat het hier om het bestandsformaat.
    Momenteel zijn alle afbeeldingen opgeslagen als JPEG bestanden, maar een moderner formaat zoals WEBP zou beter zijn om data verkeer te verminderen.

    \subsection{Efficiently encode images}
    Dit sluit wederom weer aan bij de vorige twéé punten, maar dan gaat deze echt specifiek om het compressen van de afbeeldingen.
    Het is dus nogmaals een verstandig idee om een compressie algoritme op de afbeeldingen los te laten.

    \subsection{Enable text compression}
    Alle tekst wordt kaal naar de website gestuurd door de API. Hier bevindt zich dus een verbetervoorstel, want we kunnen ook de tekst eerst met bijvoorbeeld
    {\fontfamily{courier}\selectfont gzip} encoden, naar de front-end sturen, en hem daar weel decoden naar normale tekst. Dit zou dataverkeer moeten verminderen.

    Nu ben ik het hier niet mee eens, want het gaat hier om enorm weinig tekst. Dit is misschien verstandig voor iets als Wikipedia, maar in het geval van deze website
    haalt het een paar kilobytes aan dataverkeer weg en voegt het enorm veel meer werk toe voor de client om uit te voeren.

    \subsection{Reduce unused JavaScript}
    Dit is een verbetervoorstel waar ik vrij weinig mee kan. Ik gebruik Nuxt als framework, en die neemt de taak op zich om JS te genereren.
    Ik heb dus weinig invloed op de hoeveelheid ongebruikte JS die gegenereerd wordt. Ik snap het idee, maar het is in mijn situatie niet uitvoerbaar.

    \subsection{Avoid enormous network payloads}
    Één ding waar mijn front-end een beetje tegen aan loopt is mijn back-end API. Ik heb het zo geschreven dat de API alle mogelijke relaties gelijk
    meegeeft in de API responses. Dit maakt het gemakkelijk om mee te coderen, want ik hoef maar één API call te maken, maar het genereerd gigantische
    lijsten aan data. Deze API calls worden wederom wel asynchroon uitgevoerd, dus gebruikers zien de website niet in elkaar gestapeld worden.
    Ik heb gebruik gemaakt van \textit{skeletons} als een indicator van laden. Het verschil tussen een normaal laad icoontje en een skeleton is dat een skeleton
    al de layout van de daadwerkelijke website heeft. Er verspringt dus aanzienlijk weinig.

    \section{Toegankelijkheid}
    \subsection{Image elements do not have [alt] attributes}
    Op het gebied van toegankelijkheid is het grootste minpunt het gebrek aan de {\fontfamily{courier}\selectfont alt} attributes.
    Deze worden gebruikt wanneer een afbeelding niet ingeladen kan worden, maar ook door screenreaders (applicaties die de teksten van een website uitlezen voor minderzienden).
    Het is dus wel een belangrijk iets om te hebben, en ook makkelijk te implementeren. Ik was vergeten dit te doen tijdens het bouwen van de applicatie, dus neem ik het graag mee
    als een verbetervoorstel.

    \subsection{Links do not have a discernible name}
    Nog een grote misklapper op het gebied van toegankelijkheid is het gebrek aan {\fontfamily{courier}\selectfont aria-label} attributes.
    Deze kun je toevoegen aan links binnen de website. Screenreaders lezen deze vervolgens uit om duidelijkheid te geven waar een link voor dient.
    In de meeste instanties ook een makkelijke oplossing om toe te passen, dus wederom ééntje om mee te nemen als verbetervoorstel.

    \subsection{Background and foreground colors do not have a sufficient contrast ratio}
    Mijn applicatie maakt gebruik van een donker thema. Dit houdt in dat alle kleuren aan de donkere kant van het kleurenspectrum liggen.
    Ik heb er bovendien voor gekozen om ook donkere tekst te gebruiken bij links en elementen waar ik de aandacht op wil verminderen.
    Denk bijvoorbeeld aan de links naar de gameontwikkelaars. Deze links zijn lichtgrijs op een donkergrijze achtergrond. Dit lijkt achteraf
    toch niet zo'n goed idee te zijn, want mensen met kleurenblindheid kunnen hier veel moeite mee hebben. Een idee is dus om een hoger contrast te gebruiken.

    \subsection{[html] element does not have a [lang] attribute}
    Het probleem dat Lighthouse hier identificeert is het gebrek aan een duidelijk aangegeven taal waar de webbrowser gebruik van kan maken.
    Momenteel moet de browser gokken welke taar er wordt gebruikt op de website. Webbrowsers weten inmiddels wel de juiste talen te gokken aan de hang van de teksten,
    maar het is niet netjes. De oplossing is dus om een vaste taal toe te voegen. In het geval van deze website zal dat \textit{EN} zijn.

    \section{Best practices}
    \subsection{Uses deprecated APIs}
    Lighthouse geeft aan dat één van de APIs die ik gebruik niet meer ontwikkeld wordt.
    In de toelichting geeft Lighthouse aan dat het gaat om de \textit{'Expect-CT'} header. Deze wordt door Nuxt toegevoegd, dus kan ik hier weinig aan veranderen.

    \section{SEO}
    \subsection{Document does not have a meta description}
    Wanneer u Google gebruikt herkent u misschien wel het korte stuk tekst dat onder elke link gevonden kan worden.
    Als developer heeft u de mogelijkheid om deze tekst zelf uit te schrijven en hem te verwerken in de {\fontfamily{courier}\selectfont meta} tags van de HTML.
    Wanneer een developer er voor kiest om dit niet te doen, gebruikt Google meestal de eerste aantal worden van een stuk tekst die op de pagina gevonden kan worden.
    Het handmatig schrijven van een korte omschrijving kan veel verschil maken, want het geeft de ontwikkelaar de kans om zelf een krachtige introductie te schrijven.

    \subsection{Links are not crawlable}
    Lighthouse bedoelt hiermee te zeggen dat één of meerdere links nergens naartoe leiden. In deze situatie gaat het om een ''link``
    die gevonden kan worden op kleinere resoluties. Deze dient als functie om een zoekbalk te openen. Een betere keuze is om van deze link een knop te maken,
    maar ik had gekozen een link te gebruiken omdat er een bestaand component is met allerlei styling en logica die exact had wat ik zocht, op het feit dat het een
    link is na. Om dit te verhelpen moet er een kopie van die link komen die als knop dient te functioneren.

    \chapter{Reflectie}
    De afgelopen twéé maanden heb ik veel geleerd. Ik heb geleerd om i.p.v. een ORM gewoon de standaard SQL package te gebruiken binnen Go (die ik uiteindelijk veel fijner vind),
    Nuxt onder de knie gekregen en een (naar mijn mening) mooie site opgeleverd. Is het perfect? Natuurlijk niet, zoals Lighthouse in het vorige hoofdstuk al aangaf. Maar het is
    wel mijn beste front-end werk tot nu toe, dus daar mag ik trots op zijn.

    Wel zijn er dingen die ik liever beter had gedaan. Zo heb ik de front-end niet mobile first gemaakt. Dit beketent dat ik i.p.v. het eerst designen voor mobiel en vervolgens
    opschalen naar desktop het heb ontworpen voor desktop en moeten verkleinen naar mobiel. Dat is een enorm lastige taak, vooral met de componenten waar ik mee werkte.
    Uiteindelijk is het gelukt, maar ik zou het niet nog eens op deze wijze willen doen. Een grote les geleerd dus.

    Verder heb ik mijn Go API niet uitstekend neergezet. Het grootste probleem is het gebrek aan error handling. De API stuurt eigenlijk alleen maar HTTP status codes 200 en 500 terug.
    Dit maakt het lastig voor de front-end om tegen aan te programmeren. Een ander groot probleem is dat mijn API een \textit{null pointer dereference} error krijgt wanneer er geen
    items in de database staan. Dit kan in de praktijk eigenlijk echt niet, want daarmee crasht je hele applicatie.

    Naast deze twéé punten ben ik alsnog erg tevreden over de uitwerking van dit keuzedeel. Ik zie mijzelf zeker meer Nuxt gebruiken en mobile-first ontwikkelen.
    Een geslaagd keuzedeel, als je het aan mij vraagt.

    \printbibliography
\end{document}
